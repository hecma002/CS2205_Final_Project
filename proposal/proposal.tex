\documentclass{article}
\usepackage[utf8]{vietnam}
\usepackage{tabularx}
\usepackage{geometry}
\usepackage{longtable}
\usepackage{blindtext}


\geometry{
  paper=a4paper,
  left=3cm,
  right=2cm,
  vmargin=2cm,
  includeheadfoot=true,
  headheight=30pt
}

\begin{document}
\begin{center}
    \section*{ĐỀ CƯƠNG NGHIÊN CỨU}
\end{center}

\textbf{TÊN ĐỀ TÀI: MỘT KHUÔN KHỔ HỖ TRỢ LỰA CHỌN DỊCH VỤ CỦA CÁC NHÀ CUNG CẤP ĐIỆN TOÁN ĐÁM MÂY DỰA TRÊN ĐIỂM CHUẨN CẤU HÌNH MÁY ẢO} \\

\textbf{TÊN ĐỀ TÀI TIẾNG ANH: A SUPPORTING FRAMEWORK FOR SERVICE SELECTION OF CLOUD PROVIDER BY VIRTUAL MACHINE ‘S SPEC BENCHMARK} \\

\textbf{TÓM TẮT}: Trong giai đoạn phát triển mạnh mẽ của điện toán đám mây, việc lựa chọn dịch vụ và nhà cung cấp phù hợp là một thách thức đáng kể. Số lượng nhà cung cấp rất lớn, mỗi nhà cung cấp lại cung cấp nhiều dịch vụ với mức giá và cam kết chất lượng khác nhau. Những cam kết này phức tạp và thường không dễ kiểm chứng, vì công cụ theo dõi hiệu suất thường được thiết kế bởi từng nhà cung cấp. Do đó, việc lựa chọn đòi hỏi sự hiểu biết cao về lĩnh vực này hoặc việc thuê những chuyên gia từ nhà cung cấp. Điều này đã làm giảm sự đơn giản và tiết kiệm mà điện toán đám mây mang lại. \\

CloudBench, một khuôn khổ trực quan và tập trung hỗ trợ lựa chọn dịch vụ phù hợp với nhu cầu. Thông qua điểm chuẩn trên các máy ảo Linux trên Amazon Web Service và Google Cloud Platform, CloudBench giải thích lý do khác nhau về giá cả giữa các dòng máy ảo có cùng cấu hình của một nhà cung cấp, cũng như  giữa các nhà cung cấp khác nhau. 

\end{document}

